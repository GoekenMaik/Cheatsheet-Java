\documentclass[a4paper, 8pt]{extarticle}
\usepackage[a4paper, inner=0.4cm, outer=0.4cm, top=0.3cm, bottom=0.3cm]{geometry}

\input{cheatsheetdef.tex}
\usepackage{blindtext}
\usepackage{hyperref}
\usepackage{multirow}
\usepackage{listings}
\usepackage{fontspec}
\usepackage[table]{xcolor}
\usepackage[normalem]{ulem}
\setsansfont[ItalicFont={Fira Sans Light Italic}, BoldFont={Fira Sans}, BoldItalicFont={Fira Sans Italic}]{Fira Sans Light}
\setmonofont[BoldFont={Fira Mono Medium}]{Fira Mono}
\setmainfont[ItalicFont={Fira Sans Light Italic}, BoldFont={Fira Sans}, BoldItalicFont={Fira Sans Italic}]{Fira Sans Light}
\definecolor{headcol}{HTML}{23373b}


\begin{document}
\section*{Cheatsheet Java}\vspace{-10pt}
\begin{minipage}[t]{0.25\linewidth}\vspace{0pt}
\begin{tabular}{|p{0.9\linewidth}|}

\multicolumn{1}{|c|}{\cellcolor{headcol}\color{white}Comments}\\
\\[-7pt]
Single-line Comment:
\\[-3pt]
\begin{lstlisting}[language=Java, aboveskip=-2pt,belowskip=-6pt]
 String txt = "Hello!";
 //this is a Comment
 System.out.println(txt);\end{lstlisting}
 \\
 Multi-line Comment:
 \\[-3pt]
 \begin{lstlisting}[language=Java, aboveskip=-2pt,belowskip=-6pt,mathescape]
 String txt = "Hello!";
 /*Comments will not be
 executed */
 System.out.println(txt);  
\end{lstlisting}
\\
\hline
\end{tabular}
\\[3pt]
\begin{tabular}[t]{|p{0.9\linewidth}|}

\multicolumn{1}{|c|}{\cellcolor{headcol}\color{white} Control structures}
\\
\begin{lstlisting}[language=Java, mathescape, aboveskip=-2pt,belowskip=-7pt]
if(condition1){
  /*if condition1 true, 
  execute*/
}
else if(condition2){
  /*if condition1 false and
  condition2 true, execute */
}
else{
  //if everything false, execute
}\end{lstlisting}
\\\hline
\end{tabular}
\\[3pt]
\begin{tabular}[t]{|p{0.9\linewidth}|}

\multicolumn{1}{|c|}{\cellcolor{headcol}\color{white}Loops}
\\
\begin{lstlisting}[language=Java, mathescape, aboveskip=-2pt,belowskip=-7pt]
for(int i=0; i<10; i++){
  //execute 10 times
}
while(condition){
  //execute as long as condition
}
//Do-While Loop
do{
  //execute at least once
}while(condition);
//For-Each Loop
for(List<String> s : list1){
  s.length();
}
\end{lstlisting}
\\\hline
\end{tabular}
\\[3pt]
\begin{tabular}[t]{|p{0.9\linewidth}|}

\multicolumn{1}{|c|}{\cellcolor{headcol}\color{white}Switch}
\\
\begin{lstlisting}[language=Java, mathescape, aboveskip=-2pt,belowskip=-7pt]
switch(expression){
  case 1:
    //execute if expression==1
    break;
  case 2:
    //execute if expression==2
    break;
  default:
    /*execute if expression is
     not 1 or 2 */
    break;
}
\end{lstlisting}
\\\hline
\end{tabular}
\\[3pt]
\begin{tabular}{|p{0.9\linewidth}|}
\multicolumn{1}{|c|}{\cellcolor{headcol}\color{white}Functions}\\
\begin{lstlisting}[language=Java, aboveskip=-2pt,belowskip=-7pt]
//Delaration and Implementation
<ret-type> <func-name>(<para-type> <para-name>, ...){
    // function body
    //execute
    return <expression>;
}
//Function call
<func-name>(<argument>, ...);
\end{lstlisting}
\\
\hline
\end{tabular}
\\[3pt]
\begin{tabular}{|p{0.9\linewidth}|}

\multicolumn{1}{|c|}{\cellcolor{headcol}\color{white}Operations}\\
\\[-7pt]
Arithmetic:
\\[0pt]
\hspace{3pt}\begin{tabular}{c|r}
Operation & Example\ \ \  \\\hline
+ & 3 + 5 == 8 \\
- & 7 - 2 == 5 \\
$*$ & 4 * 2 == 8 \\
/ & 7 / 2 == 3 \\
\% (Modulo) & 72 \% 10 == 2
\end{tabular}
\\[25pt]
Comparison:
\\[0pt]
\begin{tabular}{c|c|r}
Operator & Math& Example \\\hline
\textgreater & $>$ & 5 \textgreater \ 2\\
\textgreater= & $\geq$ & 5 \textgreater=\ 2\\
\textless & $<$ & 10 \textless\ 21\\
\textless= &$\leq$ & 5 \textless=\ 5\\
== & = & 5 == 5\\
!= & $\neq$ & -32 != 32 
\end{tabular}
\\[31pt]\hline
\end{tabular}
\\[3pt]
\end{minipage}\begin{minipage}[t]{0.25\linewidth}\vspace{0pt}
\begin{tabular}{|p{0.9\linewidth}|}
\multicolumn{1}{|c|}{\cellcolor{headcol}\color{white}Types}\\
\\[-7pt]
Primitive data types:
\\[-3pt]
\vspace{0pt}\begin{tabular}[t]{p{0.07\linewidth} r}
Type & Size
\\\hline
byte & 8 bit\\
short & 16 bit\\
int & 32 bit\\
long & 64 bit\\
\end {tabular}
\begin{tabular}[t]{p{0.20\linewidth} p{0.19\linewidth}}
Type & Size
\\\hline
float & 32 bit\\
double & 64 bit\\
\\[-7pt]
Type & Value\\\hline
char & 'a', 'G'\\
\multirow{2}{*}{boolean} & true, false\\
void & -
\end {tabular}
Typecasting:
\ $byte\rightarrow short\rightarrow char\rightarrow int\rightarrow long\rightarrow float\rightarrow double $
\\[110pt]
Non-Primitive data types:
\\[3pt]
\begin{tabular}[t]{p{0.15\linewidth} p{0.55\linewidth}}
Type & Value
\\\hline
String & "Hello World!"
\\
\multirow{2}{*}{Array} & int[] myNum = \{10, 20, 30, 40\};
\\
\end {tabular}
\\[8pt]\hline
\end{tabular}
\\[3pt]
\begin{tabular}{|p{0.9\linewidth}|}

\multicolumn{1}{|c|}{\cellcolor{headcol}\color{white}Declaration, Initialisation}\\
\\[-7pt]
Declaration:\ int a;\ String txt;\\
\footnotesize\textless Type\textgreater \textless \ Name\textgreater;
\\
\\[-7pt]
Initialisation:\ int b = 50; int b = a; \\
\footnotesize \textless Type\textgreater\textless Name\textgreater =\textless Literal/Variable\textgreater ;\\
\\[-7pt]
Assignment:\ a = b;\ txt = "abc";
\\[3pt]\hline
\end{tabular}
\\[3pt]

\begin{tabular}{|p{0.9\linewidth}|}
\multicolumn{1}{|c|}{\cellcolor{headcol}\color{white}Arrays}\\
\begin{lstlisting}[language=Java, aboveskip=-2pt,belowskip=-6pt]
//Declaration
<type>[] <name>;
int[] arr;
//allocation
<name> = new <type>[<size>];
arr = new int[5];
//or
<name> = {<element1>, ...};
arr = {1, 2, 3, 4, 5};
//Access
<name>[<index>];
arr[2] = 5;
\end{lstlisting}
\\\hline
\end{tabular}
\\[3pt]
\begin{tabular}{|p{0.9\linewidth}|}
\multicolumn{1}{|c|}{\cellcolor{headcol}\color{white}Strings}\\
\begin{lstlisting}[language=Java, aboveskip=-2pt,belowskip=-6pt]
/*Strings are immutable and come 
with a number of methods 
already implemented*/
//Declaration
String <name>=new String(<value>);
String helloString=new String("hello");
//or
String <name>=<value>;
String helloString="hello";
//Small Selection of useful Methods
helloString.length();
helloString.charAt(<index>);
helloString.split(" ");
\end{lstlisting}
\\\hline
\end{tabular}
\\[3pt]
\begin{tabular}{|p{0.9\linewidth}|}
\multicolumn{1}{|c|}{\cellcolor{headcol}\color{white}Collections}\\
\\[-7pt]
Common  datatypes are implemented in the 
java.util package:\\
\begin{lstlisting}[language=Java, aboveskip=-2pt,belowskip=-6pt]
import java.util.*;
/* Lists are ordered collections of 
objects, similar to arrays */
List<type><name>=new ArrayList<>();
List<String> list1=new 
ArrayList<>();
list1.add("Hello");
list1.add("World");
System.out.println(list1.get(1));
/*Sets are unordered and duplicate 
free collections of objects */
Set <type><name> = new HashSet<>();
Set<String> set1 = new HashSet<>();
set1.add("1");
set1.add("2");
set1.add("1"); //not added
System.out.println(set1);
//Output either [1, 2] or [2, 1]
//Maps let you access data via a key
Map<type1, type2><name> = new HashMap<>();
Map<Integer, String> map = new HashMap<>();
map.put(23, "foo");
map.put(28, "foo");
map.put(23, "bar"); //overwrites 23
System.out.println(map); 
//Output {23=bar, 28=foo} 
System.out.println(map.get(23)); 
//Output 'bar'
\end{lstlisting}
\\\hline
\end{tabular}
\\[3pt]
\end{minipage}\begin{minipage}[t]{0.25\linewidth}\vspace{0pt}\begin{tabular}{|p{0.9\linewidth}|}
\multicolumn{1}{|c|}{\cellcolor{headcol}\color{white}Object-Oriented Programming}\\
\vspace{-11pt}
\fontsize{8}{5}\selectfont
\begin{itemize}
\setlength{\itemindent}{-15pt}
\small
\item[] Attributes: 
\vspace{-5pt}
\begin{itemize}
\small
\setlength{\itemindent}{-15pt}
\setlength{\itemindent}{-37pt}\item[] Data (Variables)
\item[] Describes the State of the Object
\item[] Modifier always private
\end{itemize}
\item[] Methods: 
\vspace{-5pt}
\begin{itemize}
\small
\setlength{\itemindent}{-37pt}\item[] Code/Function
\item[] Changes the state of the object, 
\item[] Or interacts with other objects
\item[] Modifier mostly public
\end{itemize}
\end{itemize}
\begin{lstlisting}[language=Java, aboveskip=-2pt,belowskip=0pt]
// Defining Class
class <class-name>{
  //Attributes
  <modifier> <type> <var-name>;
  //Methods
  <modifier> <ret-type> <func-name>(<para-type> <para-name>, ...){
    // function body
  }
}
\end{lstlisting}
\begin{lstlisting}[language=Java, aboveskip=0pt,belowskip=0pt]
class Room {
  private int chairs = 4; //Attribute
  public void addChairs(int chairs)
  {
    this.chairs += chairs;
  } //Method
}\end{lstlisting}
\vspace{5pt}
\begin{lstlisting}[language=Java, aboveskip=0pt,belowskip=-6pt]
//Creating Object
<class-name> <obj-name> = 
new <class-name>();
Room kitchen = new Room();

//Accessing Attributes and Methods
<obj-name>.<var-name>; //Attribute
kitchen.chairs;

<obj-name>.<func-name>
(<argument>, ...); //Method
kitchen.addChairs(2);

/*to access members of own class 
use keyword this:*/
this.<var-name>;
this.<func-name>(<argument>, ...);
this.chairs += 5;

\end{lstlisting}
\vspace{7pt}
\fontsize{8}{5}\selectfont
\begin{minipage}[l]{1\linewidth}
\vspace{7pt}
Access modifiers\small \ to define access to an attribute
or method:\begin{itemize}
\setlength{\itemindent}{-17pt}
\item public: Anyone can access the member, default
\item private: Only the class itself can access the member
\item protected: Only the class itself and its subclasses can access the member
\end{itemize}
\vspace{2pt}
\end{minipage}
\begin{minipage}[l]{1\linewidth}
\vspace{7pt}
Constructor:\small
\begin{itemize}
\setlength{\itemindent}{-20pt}
\item[] same name as class
\item[] will get called if a new object is created
\item[] mostly used for Initialisation of attributes
\end{itemize}
\vspace{2pt}
\end{minipage}
\begin{lstlisting}[language=Java, aboveskip=0pt,belowskip=-2pt]
class <class-name> {
  public <class-name>(...){
    //constructor body
  }
  ...
}
class Student {
  public Student(String name, ...){
    this.name = name;
    ...
  }
}
\end{lstlisting}
\\\hline
\end{tabular}
\end{minipage}
\begin{minipage}[t]{0.25\linewidth}\vspace{0pt}

\begin{tabular}{|p{0.9\linewidth}|}
\multicolumn{1}{|c|}{\cellcolor{headcol}\color{white}Inheritance}\\
\begin{lstlisting}[language=Java, aboveskip=-2pt,belowskip=-5pt]
/* To give a subclass all members
of a superclass
to inherit use 'extends' keyword */
class Vehicle {
  ...
}
class Car extends Vehicle {
  ...
}
\end{lstlisting}
\\[-8pt]
\begin{lstlisting}[language=Java, aboveskip=0pt,belowskip=-7pt]
/* use 'super' to refer 
to the superclass */
class <Subclass-name> extends 
<Superclass-name> {
  public <Subclass-name>(...){
    super();
  }

  /*use @Override to replace a 
  method from the superclass */
  @Override
  public <Superclass-Method>(){
    /* calls the method
    of the superclass */
    super.<Superclass-Method>();
    //insert own code here
  }
}\end{lstlisting}
\\[3pt]\hline
\end{tabular}
\\[3pt]
\begin{tabular}{|p{0.9\linewidth}|}
\multicolumn{1}{|c|}{\cellcolor{headcol}\color{white}Abstract Classes and Inheritance}\\
\begin{lstlisting}[language=Java, aboveskip=-2pt,belowskip=-7pt]
/* Abstract classes cannot be 
instantiated and need to be 
inherited by subclasses, 
abstract functions are declarations 
of functions that have to be 
implemented in subclasses */
public abstract class <class-name> {
  //abstract method
  public abstract <ret-type> <func-name>(...);
  ...
}
\end{lstlisting}
\\[-8pt]
\begin{lstlisting}[language=Java, aboveskip=0pt,belowskip=2pt]
/* Interface is a group of 
related methods with no 
implementation. A class can 
implement multiple interfaces */
public interface <interface-name> {
  public <ret-type> <func-name>
  (...);
}

public class <class-name> implements <interface-name> {
  ...
}\end{lstlisting}
\\[3pt]\hline
\end{tabular}
\\[3pt]
\begin{tabular}{|p{0.9\linewidth}|}
\multicolumn{1}{|c|}{\cellcolor{headcol}\color{white}Static Variables, Static Functions}\\
\\[-7pt]
Static variables are variables that can be accessed from every object of the class. Only one copy of the variable exists.
Static Functions are Functions with one implementation for every object of the class. Cannot access instance variables or methods directly.
Can be accessed via Class name
\begin{lstlisting}[language=Java, aboveskip=0pt,belowskip=-7pt]
public class Test{
  public static int counter;

  public static int getCounter(){
    return counter;
  }
}

//getCounter() can be acccessed via
Test.getCounter();\end{lstlisting}
\\[3pt]\hline
\end{tabular}
\\[3pt]
\begin{tabular}{|p{0.9\linewidth}|}
\multicolumn{1}{|c|}{\cellcolor{headcol}\color{white}Generics}\\
\\[-7pt]
Generics are used to create classes
for different data types:
\\[-3pt]
\begin{lstlisting}[language=Java, aboveskip=-2pt,belowskip=-6pt]
  public class Tuple<T> {
    private T item1, item2;
    public void set(T item1, T item2) {
        this.item1 = item1;// ...
    }
    public T get(int index) {
        //return item1 or item2
    }
}
\end{lstlisting}
\\
\hline
\end{tabular}
\end{minipage}
\vspace{\fill}
\fontsize{10}{9}\selectfont
\begin{tabular}{ll}
Official Documentation: & \url{https://docs.oracle.com/en/java/javase/18/docs/api/index.html}\\

Educational: & \url{https://www.w3schools.com/java/}
\end{tabular}
\end{document}